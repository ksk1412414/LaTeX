\documentclass[tikz,border=5pt]{standalone}
\usepackage{amsmath}
\usepackage{tikz}
\usepackage{pgfplots}
\usetikzlibrary{decorations.pathmorphing,arrows.meta}
\pgfplotsset{compat=1.18}

\begin{document}
\begin{tikzpicture}[scale=1.1]

  % スリット面とスクリーン
  \draw[thick] (0,-2.2) -- (0,2.2) node[above] {スリット面};
  \draw[thick] (6,-2.2) -- (6,2.2) node[above] {スクリーン};

  % スリット幅とラベル
  \draw[very thick] (0,-0.8) -- (0,0.8);
  \draw[<->] (-0.3,0.8) -- (-0.3,-0.8) node[midway,left] {$d$};
  \draw[<->] (0.1,-0.1) -- (0.1,0.1) node[midway,right] {$w$};

  % 座標軸ラベル
  \node at (-0.4,2) {$x_P$};
  \node at (6.4,2) {$x_Q$};

  % スリットからの素元波(半円)
  \foreach \y in {-0.6, 0, 0.6} {
    \draw[dashed] (0,\y) ++(0:0.2) arc[start angle=0,end angle=180,radius=0.2];
    \draw[dashed] (0,\y) ++(0:0.4) arc[start angle=0,end angle=180,radius=0.4];
  }

  \node[right] at (0.6,0.7) {素元波};

  % スリット点Pとスクリーン点Q
  \filldraw[black] (0,-0.8) circle (0.05) node[left] {P};
  \filldraw[black] (6,0) circle (0.05) node[right] {Q};

  % 距離 L と D
  \draw[<->] (0,-1.5) -- (6,-1.5) node[midway,below] {$L$};
  \draw[dashed] (0,-0.8) -- (6,0) node[midway,below right] {$D$};

  % 回折強度分布(sinc^2)
  \begin{axis}[
      at={(6,0)},
      anchor=origin,
      axis x line=none,
      axis y line=left,
      ymin=0,
      ymax=1.2,
      width=2.4cm,
      height=4cm,
      samples=200,
      domain=-10:10,
      xtick=\empty,
      ytick=\empty,
      enlargelimits=false,
      axis line style={draw=none},
    ]
    \addplot[very thick] {(sin(deg(pi*x))/(pi*x))^2};
  \end{axis}

  \node at (4.4,2.2) {回折強度分布};

\end{tikzpicture}
\end{document}
