\documentclass{standalone}
\usepackage{tikz}
\usepackage{fontspec}
\usepackage{physics}
\setmainfont{IPAexMincho}
\usetikzlibrary{arrows.meta, angles}

\begin{document}
\begin{tikzpicture}[scale=1.5]
% ステージ(薄く、色なし)
\draw (-0.8,-0.15) rectangle (0.8,-0.05);
\node[right] at (-0.5,-0.3) {ステージ};

% 試料(さらに薄く)
\draw (-0.5,0) rectangle (0.5,-0.05);
\draw (0.5,0) -- (0.7,0.11);
\node[right] at (0.6,0.11) {試料};

% X線源(左上、入射線に垂直に設置)
\draw[rotate around={45:(-1.5,1.5)}] (-1.8,1.2) rectangle (-1.2,1.8);
\node at (-1.5,2.1) {X線源};
\draw [->] (-1.1,1.1) arc[start angle=-45, end angle=-75,radius=0.4];
\draw [->] (-1.1,1.1) arc[start angle=-45, end angle=-15,radius=0.4];

% 入射ビーム(斜め下に向かう)
\draw[->] (-1.5,1.5) -- (0,0);

% 検出器(右上、反射線に垂直に設置)
\draw[rotate around={-45:(1.5,1.5)}] (1.2,1.2) rectangle (1.8,1.8);
\node at (1.5,2.1) {検出器};
\draw [->] (1.1,1.1) arc[start angle=-135, end angle=-165,radius=0.4];
\draw [->] (1.1,1.1) arc[start angle=-135, end angle=-105,radius=0.4];

% 反射ビーム(試料表面で反射、斜め上へ)
\draw[dashed,->] (0,0) -- (1.5,1.5);

% 入射角θ
\draw[dashed] (-0.4,0) arc[start angle=180, end angle=135, radius=0.4];
\node at (-0.6,0.2) {$\theta$};

% 反射角θ
\draw[dashed] (0.4,0) arc[start angle=0, end angle=45, radius=0.4];
\node at (0.6,0.2) {$\theta$};

\end{tikzpicture}
\end{document}
